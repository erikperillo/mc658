\documentclass[7pt]{article}
\usepackage{graphicx}
\usepackage{float}
\usepackage{amsmath}
\usepackage{amsfonts}
\usepackage[brazilian]{babel}
\usepackage[utf8]{inputenc}
\usepackage[backend=biber]{biblatex}
\usepackage{csquotes}
%\usepackage{docmute}
\usepackage{array}
\usepackage{multicol}
\usepackage[bindingoffset=0.0in,%
            left=0.9in,right=0.9in,top=1.0in,bottom=1.0in,%
            voffset=0in,footskip=1.5in]{geometry}
\usepackage{algorithm}
\usepackage{titling}
\usepackage[noend]{algpseudocode}
\usepackage[T1]{fontenc}

\addbibresource{report.bib}

\newcommand{\fromeng}[1]{\footnote{do inglês: \textit{#1}}}
\newcommand{\tit}[1]{\textit{#1}}
\newcommand{\tbf}[1]{\textbf{#1}}
\newcommand{\ttt}[1]{\texttt{#1}}

\begin{document}

\author{\small Erik Perillo, RA135582}
\date{\small \today}
\title{\vspace{-2cm}%
	{\small MC658 - Projeto e análise de algoritmos III - Unicamp}\\
    {\Large \tbf{Laboratório 5 -- s-t Caminhos com Coleta de Prêmios}}}
%\subtitle{Lel}
\maketitle\vspace{-0.5cm}

\makeatletter
\def\BState{\State\hskip-\ALG@thistlm}
\makeatother

\begin{multicols}{2}

\section{Problema}
O problema consiste em, dado um grafo orientado $G = (N, A)$, encontrar o
melhor caminho $P = (N_P, A_P)$ de $s$ a $t$ que maximize uma certa
função de custo.

\section{Heurística}
De modo a obter um bom \tit{lower bound} para o problema, foi desenvolvida
uma heurística baseada na metaheurística
\tit{GRASP - Greedy Randomized Adaptive Search Procedure}.
Os pseudo-códigos são descritos a seguir.
\end{multicols}

\begin{algorithm}
\caption{}\label{grasp}
\begin{algorithmic}[1]
    \Procedure{GRASP-HEURISTICA}{G, s, t, prize, cost}
    \State $S \gets \emptyset$
    \State $S_+ \gets \emptyset$
    \While {not $timeout$ and not $limit\_its$ and not $stuck\_in\_local\_opt$}
        \State $S \gets$ RANDOM-GREEDY-SOL-DFS(S, s, t, prize, cost)
        \State $S \gets$ LOCAL-SEARCH(S, G, prize, cost)
        \If{$cost(S) \ge cost(S_+)$}
            \State $S_+ \gets S$
        \EndIf
	\EndFor
    \Return $S_+$
    \EndProcedure
\end{algorithmic}
\end{algorithm}

\begin{algorithm}
\caption{}\label{random-dfs}
\begin{algorithmic}[1]
    \Procedure{RANDOM-GREEDY-SOL-DFS}{S, s, t, prize, cost}
    \State set $s$ as visited
    \If{$s = t$}
        \State return TRUE
    \EndIf
    \State $A \gets Adj(s)$
    \While {$A \neq \emptyset$}
        \State select $(s, v)$ with probability proportional to
            $prize(s) - cost((s,v))$
        \State $A \gets A - \{(s, v)\}$
        \If{not visited($v$) and RANDOM-GREEDY-SOL-DFS(S, v, t, prize, cost)}
            \State $S \gets S \cup {(s, v)}$
            \State \Return TRUE
        \EndIf
    \EndWhile
    \State \Return FALSE
    \EndProcedure
\end{algorithmic}
\end{algorithm}

\begin{algorithm}
\caption{}\label{local}
\begin{algorithmic}[1]
    \Procedure{LOCAL-SEARCH}{S, G, prize, cost}
    \While {not $timeout$ and not $limit\_its$ and not $stuck\_in\_local\_opt$}
    \For {$(u, v) \in S$}
        \For{$(u, w) \in A(G)$ and $(w, v) \in A(G)$ with $w \notin S$}
            \If{$prize(w) - cost((u,w),(w,v)) \ge -cost(u, v)$}
                \State $S \gets S \cup{\{(u, w), (w, v)\}} - \{(u, v)\}$
            \EndIf
        \EndFor
	\EndFor
    \EndWhile
    \Return $S$
    \EndProcedure
\end{algorithmic}
\end{algorithm}

\begin{multicols}{2}
O algoritmo~\ref{grasp} é a metaheurística principal:
enquanto o critério de parada não é verdade,
ele busca uma solução gulosa-probabilística inicial
e a refina com busca local.
O algoritmo~\ref{random-dfs} explica a ideia da busca probabilística gulosa:
Faz-se o caminho em DFS de $s$ a $t$, escolhendo nós/arcos com probabilidades
proporcionais a seus custos: quanto maior, maior a chance de escolher.
O algoritmo~\ref{local} explica a ideia da busca local:
para cada arco da solução aleatória-gulosa, ele procura outros dois arcos
que formam um caminho entre os nós do arco original mas com custo melhor.
Ele repete isso no máximo 3 vezes.
A heurística obteve resultados sempre muito próximos do \tit{upper bound}.

\section{Solução exata com Programação Linear Inteira}
Seja $x_v$ a variável binária que é $1$ se o nó $v$ está na solução e
$0$ caso contrário.
Seja $x_a$ a variável binária que é $1$ se o arco $a$ está na solução e
$0$ caso contrário.
Seja $\pi_v$ os prêmios de cada nó $v$ e $c_a$ os custos de cada arco $a$.

Queremos encontrar o caminho de $s$ a $t$ $P = (N_P, A_P)$ que maximize:
$$\ttt{max} z = \sum\nolimits_{v \in N}x_v \pi_v
    - \sum\nolimits_{a \in A}x_a c_a$$
Sujeito a:
\begin{itemize}
    \item Do nó $s$, só tem um arco de saída e nenhum de entrada:
        $$\sum\nolimits_{x_a \in \delta_+(s)}x_a = 1$$
        $$\sum\nolimits_{x_a \in \delta_-(s)}x_a = 0$$

    \item Do nó $t$, só tem um arco de entrada e nenhum de saída:
        $$\sum\nolimits_{x_a \in \delta_+(t)}x_a = 0$$
        $$\sum\nolimits_{x_a \in \delta_-(t)}x_a = 1$$

    \item Um nó que não seja $s$ e $t$ é selecionado se e somente se ele tem
        exatamente um arco de entrada e um de saída:
        $$\sum\nolimits_{x_a \in \delta_+(v)}x_a = x_v, ~\forall~v \in N -
            \{s, t\}$$
        $$\sum\nolimits_{x_a \in \delta_-(v)}x_a = x_v, ~\forall~v \in N -
            \{s, t\}$$

    \item Restrição de integralidade:
        $$x_a \in \{0, 1\}, ~\forall~x_a \in A$$
        $$x_v \in \{0, 1\}, ~\forall~x_v \in N$$
\end{itemize}

\section{Implementação}
A implementação foi feita com o uso do gurobi (versão \ttt{7.0.2}) e lemon.

\section{Avaliação do modelo proposto}
Na tabela a seguir, temos os valores para alguns arquivos de entrada.
O algoritmo exato recebia o \tit{lower bound} obtido pela heurística
como \tit{cutoff}.
A heurística mostrou-se muito eficiente, obtendo \tit{lower bounds}
muito próximos do valor ótimo.
O valor ótimo foi obtido em quase todos os casos.
No caso em que o OPT não foi obtido, a heurística obteve um
\tit{lower bound} muito próximo do valor ótimo (3.00684e+09).
O arquivo \ttt{100000\_700000.in} não pôde ser executado na máquina dos
testes pois ela não tinha memória suficiente.
\end{multicols}

\begin{table}[H]
\caption{Resultados de execução com lower bounds (LB) e upper bounds (UB)
obtidos.}
\centering
\begin{tabular}{|c|c|c|c|c|}
\hline
entrada & LB & UB & OPT obtido? & tempo limite (s)\\
\hline
3\_2.in & 6 & 13 & sim & 1\\
\hline
30\_600\_1.in & $9.33478\times10^6$ & $9.33510\times10^6$ & sim & 1\\
\hline
100\_5000\_1.in & $2.86350\times10^7$ & $2.86361\times10^7$ & sim & 1\\
\hline
1000\_20000\_1.in & $3.07347\times10^8$ & $3.07367\times10^8$ & sim & 1\\
\hline
10000\_9000000\_1.in & $3.00666\times10^9$ & $3.00666\times10^9$ & não & 60\\
\hline
\end{tabular}
\end{table}


\end{document}
