\documentclass[7pt]{article}
\usepackage{graphicx}
\usepackage{float}
\usepackage{amsmath}
\usepackage{amsfonts}
\usepackage[brazilian]{babel}
\usepackage[utf8]{inputenc}
\usepackage[backend=biber]{biblatex}
\usepackage{csquotes}
%\usepackage{docmute}
\usepackage{array}
\usepackage{multicol}
\usepackage[bindingoffset=0.0in,%
            left=0.9in,right=0.9in,top=0.5in,bottom=0.5in,%
            voffset=0in,footskip=1.5in]{geometry}
\usepackage{algorithm}
\usepackage{titling}
\usepackage[noend]{algpseudocode}
\usepackage[T1]{fontenc}

\addbibresource{report.bib}

\newcommand{\fromeng}[1]{\footnote{do inglês: \textit{#1}}}
\newcommand{\tit}[1]{\textit{#1}}
\newcommand{\tbf}[1]{\textbf{#1}}
\newcommand{\ttt}[1]{\texttt{#1}}

\newcolumntype{C}[1]{>{\centering\let\newline\\\arraybackslash\hspace{0pt}}m{#1}}

%\setlength{\intextsep}{1.5pt}
%\setlength{\textfloatsep}{-1in}

\begin{document}

%\newgeometry{margin=1in}

\author{Erik Perillo, RA135582\\
        Kelvin Ronny, RA138645}
\date{\today}
\title{\vspace{-2cm}%
	{\small MC658 - Projeto e análise de algoritmos III - Unicamp}\\
    {\Large Laboratório 2 - Backtracking e Branch-and-Bound}}
%\subtitle{Lel}
\maketitle\vspace{-0.5cm}
\posttitle{\par\end{center}}

\makeatletter
\def\BState{\State\hskip-\ALG@thistlm}
\makeatother

\begin{multicols}{2}

\section{Introdução}
O desafio do Laboratório é resolver o problema do empacotamento com divisórias
por meio de algoritmos de \tit{backtracking} e \tit{branch and bound},
analisando quantitativamente seus desempenhos.

\section{Backtracking}
\tit{backtrack\_rec} é a função que faz recursão dentre as possibilidades de alocação de usuários na banda. Para isso, primeiro simulamos o que acontece se tentarmos colocar mais um usuário na banda. Então, usamos a função validator para validar a possibilidade.
Se a possibilidade vale, testamos um backtracking para ela.
Senão, apenas tentamos um backtracking sem o usuario, dada a solução candidata melhor até o momento.

\section{Branch and Bound}
A técnica branch and bound ordena todos os pacotes enfileirados e monta uma
árvore com ramificação binária composta pela escolha (ou não) de um certo item.

A ordenação é feita pelo valor relativo, do maior para o menor.
O critério de poda usa dos seguintes fatos:
Seja $K$ o conjunto de classes presentes na solução atual.
Seja $r_K$ o número de pacotes cuja classe pertence ao conjunto $K$ mas não
estão na solução atual.
A melhor situação possível é que todos os $r_K$ primeiros pacotes (ordenados
por valor relativo) ainda não presentes na solução tenham classe no conjunto
$K$, pois isso evita a necessidade de uma divisória nova.
Após usar todos esses pacotes, o melhor que pode acontecer é que os pacotes
restantes, em ordem de valor relativo, pertençam respectivamente às
classes $c_1$, $c_2$, ..., $c_k$, sendo o número de elementos da classe
$c_1$ maior que o da $c_2$ e assim por diante.
Esse cenário garante o menor uso de divisórias possível.
Assim, o critério de pode assume esse cenário ideal e calcula o valor
máximo que pode ser obtido pela ramificação.
Se este ainda for menor que o atual máximo, não faz sentido continuar a
busca naquele ramo e então é feita a poda.

\ttt{bnb} implementa o algoritmo branch and bound. A rotina ordena os
elementos por valor relativo e então chama \ttt{\_bnb}, rotina recursiva que
faz a busca de fato dos valores.

\section{Tempos}
\begin{table}[H]
\centering
\setlength{\tabcolsep}{.16667em}
\caption{Branch and bound para arquivos de teste padrão.}
\begin{tabular}{|c|c|c|c|c|}
    \hline
    arquivo & t\. limite (s) & valor & tempo (s) & timeout\\
    \hline
    1000\_50\_800.in & 15 & 2177 & 16.012 & 1\\
    \hline
    100\_5\_500.in & 15 & 763 & 0.003 & 0\\
    \hline
    5\_2\_20.in & 15 & 17 & 0.002 & 0\\
    \hline
\end{tabular}
\end{table}

Nota-se que foi alcançado o valor ótimo para os três casos e os dois menores
terminaram quase instantaneamente.
Foram gerados mais valores em um intervalo mais amplo para melhor compreensão
do algoritmo:
\end{multicols}

\begin{table}[H]
\centering
\setlength{\tabcolsep}{.16667em}
\caption{Branch and bound para diversos arquivos de teste.}
\begin{tabular}{|c|c|c|c|c|}
    \hline
    arquivo & t\. limite (s) & valor & tempo (s) & timeout\\
    \hline
    10\_1\_3.in & 15 & 27 & 0.003 & 0\\
    \hline
    10\_2\_7.in & 15 & 0 & 0.003 & 0\\
    \hline
    10\_5\_3.in & 15 & 21 & 0.003 & 0\\
    \hline
    50\_13\_20.in & 15 & 75 & 0.002 & 0\\
    \hline
    50\_6\_30.in & 15 & 121 & 0.003 & 0\\
    \hline
    50\_8\_32.in & 15 & 94 & 0.003 & 0\\
    \hline
    100\_16\_79.in & 15 & 178 & 0.004 & 0\\
    \hline
    100\_26\_33.in & 15 & 104 & 0.002 & 0\\
    \hline
    100\_45\_61.in & 15 & 110 & 0.003 & 0\\
    \hline
    200\_11\_88.in & 15 & 338 & 0.006 & 0\\
    \hline
    200\_1\_96.in & 15 & 544 & 0.004 & 0\\
    \hline
    200\_78\_68.in & 15 & 183 & 0.004 & 0\\
    \hline
    300\_114\_206.in & 15 & 405 & 0.025 & 0\\
    \hline
    300\_71\_222.in & 15 & 446 & 0.029 & 0\\
    \hline
    300\_97\_276.in & 15 & 529 & 0.055 & 0\\
    \hline
    350\_124\_302.in & 15 & 464 & 0.028 & 0\\
    \hline
    350\_68\_134.in & 15 & 298 & 0.009 & 0\\
    \hline
    350\_98\_254.in & 15 & 447 & 0.037 & 0\\
    \hline
    400\_192\_267.in & 15 & 458 & 0.067 & 0\\
    \hline
    400\_29\_329.in & 15 & 916 & 0.079 & 0\\
    \hline
    400\_51\_205.in & 15 & 556 & 0.055 & 0\\
    \hline
    450\_192\_250.in & 15 & 523 & 0.084 & 0\\
    \hline
    450\_70\_354.in & 15 & 847 & 2.380 & 0\\
    \hline
    450\_81\_308.in & 15 & 612 & 0.375 & 0\\
    \hline
    500\_201\_272.in & 15 & 513 & 0.151 & 0\\
    \hline
    500\_204\_280.in & 15 & 609 & 0.143 & 0\\
    \hline
    500\_7\_434.in & 15 & 1566 & 0.005 & 0\\
    \hline
    550\_173\_341.in & 15 & 574 & 0.703 & 0\\
    \hline
    550\_236\_493.in & 15 & 769 & 2.070 & 0\\
    \hline
    550\_82\_483.in & 15 & 1023 & 16.004 & 1\\
    \hline
    600\_124\_486.in & 15 & 906 & 4.761 & 0\\
    \hline
    600\_125\_574.in & 15 & 1064 & 16.014 & 1\\
    \hline
    600\_259\_480.in & 15 & 775 & 6.515 & 0\\
    \hline
    700\_104\_423.in & 15 & 991 & 8.519 & 0\\
    \hline
    700\_272\_529.in & 15 & 802 & 4.737 & 0\\
    \hline
    700\_321\_353.in & 15 & 770 & 0.911 & 0\\
    \hline
    800\_265\_536.in & 15 & 995 & 16.003 & 1\\
    \hline
    800\_271\_532.in & 15 & 792 & 8.380 & 0\\
    \hline
    800\_298\_642.in & 15 & 1014 & 16.035 & 1\\
    \hline
    900\_225\_564.in & 15 & 1067 & 16.007 & 1\\
    \hline
    900\_322\_386.in & 15 & 721 & 5.452 & 0\\
    \hline
    900\_396\_476.in & 15 & 951 & 16.003 & 1\\
    \hline
    1000\_129\_355.in & 15 & 852 & 4.094 & 0\\
    \hline
    1000\_458\_557.in & 15 & 1022 & 16.029 & 1\\
    \hline
    1000\_72\_967.in & 15 & 2241 & 16.011 & 1\\
    \hline
\end{tabular}
\end{table}
\end{document}
