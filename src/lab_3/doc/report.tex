\documentclass[7pt]{article}
\usepackage{graphicx}
\usepackage{float}
\usepackage{amsmath}
\usepackage{amsfonts}
\usepackage[brazilian]{babel}
\usepackage[utf8]{inputenc}
\usepackage[backend=biber]{biblatex}
\usepackage{csquotes}
%\usepackage{docmute}
\usepackage{array}
\usepackage{multicol}
\usepackage[bindingoffset=0.0in,%
            left=0.9in,right=0.9in,top=1.0in,bottom=1.0in,%
            voffset=0in,footskip=1.5in]{geometry}
\usepackage{algorithm}
\usepackage{titling}
\usepackage[noend]{algpseudocode}
\usepackage[T1]{fontenc}

\addbibresource{report.bib}

\newcommand{\fromeng}[1]{\footnote{do inglês: \textit{#1}}}
\newcommand{\tit}[1]{\textit{#1}}
\newcommand{\tbf}[1]{\textbf{#1}}
\newcommand{\ttt}[1]{\texttt{#1}}

\begin{document}

\author{\small Erik Perillo, RA135582}
\date{\small \today}
\title{\vspace{-2cm}%
	{\small MC658 - Projeto e análise de algoritmos III - Unicamp}\\
    {\Large \tbf{Laboratório 3 - Programação Linear}}}
%\subtitle{Lel}
\maketitle\vspace{-0.5cm}

\begin{multicols}{2}

\section{Formulação do Problema}
O problema consiste em encontrar a melhor alocação possível de recursos entre
roteadores e servidores.

Para isso, considere um problema com
um conjunto de terminais $T$ e tamanho $|T| = n_1$
com requerimentos de banda $r_t$ para todo $t \in T$,
um conjunto de roteadores $R$ e tamanho $|R| = n_2$
com limites de banda $l_r$ para todo $r \in R$ e
um conjunto de possíveis conexões $X$ com tamanho $|X| = m$
com custo por unidade de banda $c_x$ para todo $x \in X$.
Pode-se imaginar a instância como um grafo não direcionado
$G = (V, E)$, com $V = T \cup R$ e $E = X$.

Queremos encontrar os valores $x \in X$ de modo a
minimizar o custo total das ligações:
$$\ttt{min} z = \sum\nolimits_{x \in X}xc_x$$
Sujeito a:
\begin{itemize}
    \item Cada terminal deve receber pelo menos uma certa quantidade de banda:
        $$\sum\nolimits_{x \in Adj(t)}x \ge r_t, ~\forall~t \in T$$

    \item Cada roteador pode fornecer no máximo uma certa quantidade de banda:
        $$\sum\nolimits_{x \in Adj(r)}x \le l_t, ~\forall~r \in R$$

    \item Os valores são não-negativos:
        $$x \ge 0, ~\forall~x \in X$$
\end{itemize}

\section{Implementação}
A implementação foi feita com o uso do gurobi (versão \ttt{7.0.2}) e lemon.

\section{Avaliação do modelo proposto}
O modelo implementado foi testado primeiramente em arquivos de entrada
fornecidos. Os nomes do arquivo estão no formato
\ttt{<num\_terminais>\_<num\_roteadores>.in}.
Nota-se que, para um limite de 15 segundos, o programa chegou à solução ótima
para todas as entradas com folga de tempo, como mostra a tabela a seguir.

\begin{table}[H]
\caption{Resultados de execução com lower bounds (LB) e upper bounds (UB)
obtidos.}
\centering
\begin{tabular}{|c|c|c|c|c|}
\hline
entrada & LB & UB & OPT obtido? & tempo limite (s)\\
\hline
3\_2.in & 6 & 13 & sim & 1\\
\hline
30\_600\_1.in & $9.33478\times10^6$ & $9.33510\times10^6$ & sim & 1\\
\hline
100\_5000\_1.in & $2.86350\times10^7$ & $2.86361\times10^7$ & sim & 1\\
\hline
1000\_20000\_1.in & $3.07347\times10^8$ & $3.07367\times10^8$ & sim & 1\\
\hline
10000\_9000000\_1.in & $3.00666\times10^9$ & $3.00666\times10^9$ & não & 60\\
\hline
\end{tabular}
\end{table}


Foram geradas outras entradas com tamanhos entre 1000 e 2000, os resultados
encontram-se na tabela a seguir.

\begin{table}[H]
\caption{Desempenho para entradas geradas.}
\centering
\begin{tabular}{|c|c|c|c|c|}
\hline
arquivo & tMax(s) & custo sol & tempo(s) & timeout\\
\hline
659\_692.in & 15 & 14016 & 1.05 & 0\\
\hline
785\_662.in & 15 & 30290 & 0.73 & 0\\
\hline
828\_887.in & 15 & 18472 & 1.02 & 0\\
\hline
892\_772.in & 15 & 17972 & 2.03 & 0\\
\hline
927\_856.in & 15 & 20081 & 1.05 & 0\\
\hline
942\_845.in & 15 & 19472 & 1.98 & 0\\
\hline
761\_752.in & 15 & 16008 & 1.14 & 0\\
\hline
1123\_1061.in & 15 & 31899 & 1.44 & 0\\
\hline
1412\_1126.in & 15 & 29103 & 3.34 & 0\\
\hline
1613\_1551.in & 15 & 32894 & 4.67 & 0\\
\hline
1742\_1508.in & 15 & 35983 & 6.35 & 0\\
\hline
1893\_1391.in & 15 & 42583 & 4.01 & 0\\
\hline
1935\_1408.in & 15 & 39937 & 9.81 & 0\\
\hline
2001\_1940.in & 15 & 41242 & 9.04 & 0\\
\hline
\end{tabular}
\end{table}


\end{multicols}

\end{document}
